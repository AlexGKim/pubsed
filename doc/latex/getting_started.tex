
\chapter{Getting Started}

\section{Building the Code}


To compile \sedona\ you will need a C++ compiler with MPI, the gsl library, hdf5, and the lua scripting language.
(Many systems will already have all but lua already installed, and can be simply loaded using module load.)

\begin{enumerate}


\item Set the environment variable {\tt SEDONA\_HOME} to point to the base directory of sedona. In bash, for example, you
can add to your .bash profile the line: 
\begin{verbatim}
export SEDONA_HOME=/Users/kasen/sedona6/
\end{verbatim}


\item Install lua. Source code can be found in the {\tt src/external/} directory

\item Download and install the Gnu Scientific Library (gsl) available at https://www.gnu.org/software/gsl/

\item Download and install hdf5 from https://support.hdfgroup.org/downloads/index.html

\item Go to the {\tt src} directory and edit the make.inc file so that {\tt CXX} is your C++ compiler with options, and the variables point to the  location of the installed libraries. Some make.inc files are already included for common systems (e.g., {\tt make.inc.nersc}) and can simply be copied over to {\tt make.inc}.

\item Type {\tt make}.

\item If compilation is successful, the executable file \executable\ will appear in the src/ directory. Copy this to the directory where you would like to run the code.


\end{enumerate}


Python is currently used for plotting and testing scripts, but is not needed to build and run \sedona\ itself (NB: Python may
eventually replace lua for parameter files). Python 2.7 is being used. On linux to get the necessary packages try:

\begin{verbatim}
sudo apt-get install python-numpy python-scipy python-matplotlib python h5py
\end{verbatim}



\subsection{Running the Code}

To get started, you can try running a simple transport calculation. Copy the \executable\ executable  in the {\tt examples/simple\_examples/spherical\_lightbulb/1D} directory. 

To run the code type 
\begin{Verbatim}
./sedona6.ex param.lua
\end{Verbatim}
 
where {\tt param.lua} is the file name of the run time parameter file (if no file name is given, the name {\tt param.lua} is assumed).  The parameter file must point to three files


\begin{enumerate}
	\item A model file  (an ascii .mod file for 1D runs or a hdf5 file for multi-D) giving the physical conditions (e.g., density, composition) of the setup. The location is specified 
	\item An atomic data file in hdf5 format. Such files are available in the {\tt data/} directory, with data compiled from various sources.
	\item A  defaults file, giving the default settings for all runtime parameters. The standard is {\tt defaults/sedona\_defaults.lau}, although users can point to their own modified defaults file.
\end{enumerate}

The code generates several files. The {\tt plt\_.?????.h5} files contain data in hdf5 format describing the grid properties (e.g., density, temperature, radiation field). If hdf5 tools are installed, type
\begin{Verbatim}
h5ls plt_00001.h5
\end{Verbatim}
to see the file contents. For 1D models, a {\tt plt\_?????.dat} gives some of this data in ascii format.

The code also generates a {\tt spectrum\_?????.h5} file giving the output spectrum.

You will find in the {\tt tools/} directory a python library file {\tt sedona.py} that provides functions to read and analyze the data in the plot and spectrum files (NB: this needs to be developed)


