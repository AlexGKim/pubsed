\documentclass[11pt,letterpaper]{article}

\usepackage{color}
\usepackage{amsmath}
\usepackage{amssymb}
\usepackage{hyperref}
\usepackage{seqsplit}
\usepackage{fullpage}
\usepackage{dirtree}
\usepackage[normalem]{ulem}


\newcommand{\bhead}[1]{\medskip \noindent {\bf #1}}
\newcommand{\test}[2]{\medskip \noindent   {\bf Test  \#\textsf{\arabic{#1}}} \addtocounter{#1}{1}- {\it {#2}}:}

 
% make the MarginPars look pretty
\setlength{\marginparwidth}{0.75in}
\newcommand{\MarginPar}[1]{\marginpar{\vskip-\baselineskip\raggedright\tiny\sffamily
\hrule\smallskip{\color{red}#1}\par\smallskip\hrule}}

\begin{document}

\title{SedonaBox User's Guide}
\maketitle

\section{Introduction}

\section{Getting Started}

\subsection{Getting the Code}

\subsection{Building the Code}


\subsection{Running the Code}

%To run the executable you just built, you will need the data for
%computing opacities ({\tt atoms/} and {\tt fuzz/}), an {\tt inputs}
%gile that provides runtime parameters, a {\tt probin} file specifying
%the initial model, and a model file ({\tt w7.boxmod} in this case).

%To run with 2 MPI processors and 2 threads, you do something like

%\noindent {\tt OMP\_NUM\_THREADS=2 mpiexec -n 2
 % ./Sedona1d.Linux.g++.gfortran.MPI.OMP.ex inputs.1d-regt}


\section{Input Data}

Atomic data format

\section{Output}

\section{Runtime Parameters}

\section{Test Problems}

\subsection{Core Into Vacuum}

\bhead{Setup:} A spherical inner boundary emits blackbody radiation into an extremely low density
medium, with optical depth so low it is essentially vacuum.

\newcounter{tc}
\setcounter{tc}{1}
\test{tc}{Emergent Spectrum} This should be a blackbody at the input inner core temperature. Tests general sampling of 

\test{tc}{Radiation Temperature Structure}. The radiation field outside a spherical emitter should be given by the dilution factor
\begin{equation}
J = \frac{1}{2} \left[ 1 - \sqrt{ 1 - R_0^2/r^2} \right]
\end{equation}

\test{tc}{Non-LTE level populations}





\end{document}
