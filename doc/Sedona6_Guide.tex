\documentclass[11pt,letterpaper]{article}

\usepackage{color}
\usepackage{amsmath}
\usepackage{amssymb}
\usepackage{hyperref}
\usepackage{seqsplit}
\usepackage{fullpage}
\usepackage{dirtree}
\usepackage[normalem]{ulem}
\usepackage{array}


\newcommand{\bhead}[1]{\medskip \noindent {\bf #1}}
\newcommand{\test}[2]{\medskip \noindent   {\bf Test  \#\textsf{\arabic{#1}}} \addtocounter{#1}{1}- {\it {#2}}:}

 
% make the MarginPars look pretty
\setlength{\marginparwidth}{0.75in}
\newcommand{\MarginPar}[1]{\marginpar{\vskip-\baselineskip\raggedright\tiny\sffamily
\hrule\smallskip{\color{red}#1}\par\smallskip\hrule}}

\begin{document}

\title{Sedona6 User's Guide}
\maketitle

\section{Introduction}

\section{Getting Started}

\subsection{Getting the Code}

\subsection{Building the Code}


\subsection{Running the Code}




\section{Input Data}

Atomic data format

\section{Model Parameters}

Parameters are in the param.lua file. This is lua script, where you can specify variables and even functions inside the param file.

Defaults for all parameters are given in a defaults.lua file. 

\subsection{Time-stepping Parameters}

\begin {table}[h]
\begin{tabular}{| >{\bf}l | p{12cm} |  }
\hline
\hline
  tstep\_max\_steps    &  integer number of time steps to take before exiting before stopping \\ \hline
     tstep\_time\_start    & start time (in seconds) \\ \hline
   tstep\_time\_stop    & stop time (in seconds) \\ \hline
  tstep\_max\_dt        & maximum size of a timestep (in seconds)  \\ \hline
 tstep\_min\_dt          & minimum size of a timestep (in seconds)\\ \hline
 tstep\_max\_delta   &  maximum fractional size of a timestep (multiply this by 
 the current time to get the maximum timestep). \\ \hline
\hline
\end{tabular}
\end{table}

Note that if
\subsection{Transport Parameters}


\begin {table}[h]
\begin{tabular}{| >{\bf}l | p{10cm} |  }
\hline
\hline
transport\_nu\_grid  & frequency grid to calculate opacities/emissivites. In the format of {nu\_start, nu\_stop, nu\_delta} \\ \hline
transport\_radiative\_equilibrium  & = 0 or 1. If 1, determine gas temperature after each time step from radiative equilibrium, i.e., heating equals radiative cooling \\ \hline
transport\_steady\_iterate    & = integer. Do not step in time, rather iterate the radiation transport (in steady state) the number of times given. \\ \hline
\hline
\end{tabular}
\end{table}


\newpage


\section{Output}

\bhead{Spectrum files:}

\bhead{Ray files:}

\bhead{Grid files:} 


\bhead{Level.hdf5 Files:}
These files continue detailed information about the ionization and excitation state for all species.




\section{Test Problems}

\subsection{Core Into Vacuum}

\bhead{Setup:} A spherical inner boundary emits blackbody radiation into an extremely low density
medium, with optical depth so low it is essentially vacuum.

\newcounter{tc}
\setcounter{tc}{1}
\test{tc}{Emergent Spectrum} This should be a blackbody at the input inner core temperature. Tests general sampling of 

\test{tc}{Radiation Temperature Structure}. The radiation field outside a spherical emitter should be given by the dilution factor
\begin{equation}
J = \frac{1}{2} \left[ 1 - \sqrt{ 1 - R_0^2/r^2} \right]
\end{equation}

\test{tc}{Non-LTE level populations}


\section{Physics}

\subsection{Bound-Bound}

The extinction coefficient, corrected for stimulated emission, is calculated as (Rutten eq 2.62)
\begin{equation}
\alpha_{\rm bb}(\nu) = \frac{h \nu}{4 \pi} n_l B_{\rm lu}  \phi(\nu) \left[ 1 - \frac{n_u g_l}{n_l g_u} \right]
\end{equation}
which assumes the absorption profile is the same as the emission (complete redistribution).
Here $B_{\rm lu}$ can't be a constant, so we take
\begin{equation}
B_{\rm lu} = \frac{g_u}{g_l} \frac{c^2}{2 h \nu^3} A_{\rm ul}
\end{equation}
So the extinction becomes
\begin{equation}
\alpha_{\rm bb}(\nu) =  \frac{1}{8 \pi}  \frac{g_u}{g_l} \frac{c^2}{\nu^2} n_l A_{\rm ul}  \phi(\nu) \left[ 1 - \frac{n_u g_l}{n_l g_u} \right]
\end{equation}
The emissivity is (Rutten eq. 2.69)
\begin{equation}
j_{\rm \nu, bb}(\nu) = \frac{h \nu}{4 \pi} n_u A_{\rm ul} \phi(\nu)
\end{equation}
There is still an issue here, the source function is then
\begin{equation}
S_\nu = j_\nu/\alpha_\nu = \frac{2 h \nu^3}{c^2} (e^{h \nu_0/kT} - 1)^{-1}
\end{equation}
which is not a blackbody because of the $\nu_0$, not $\nu$ in the exponential.  The resolution here is subtle, if I recall...






\end{document}
